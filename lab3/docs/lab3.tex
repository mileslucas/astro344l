\documentclass[%
aip,
jmp,
reprint,
floatfix
]{revtex4-1}

\usepackage{graphicx}% Include figure files
\usepackage{dcolumn}% Align table columns on decimal point
\usepackage{bm}% bold math
\usepackage{float}
\usepackage{siunitx}
\usepackage{hyperref}
\usepackage{listings}
%\usepackage[style=phys]{biblatex}
%\addbibresource{refs.bib}
\usepackage{ gensymb }
%\usepackage[mathlines]{lineno}% Enable numbering of text and display math
%\linenumbers\relax % Commence numbering lines
\usepackage{color}

\definecolor{dkgreen}{rgb}{0,0.6,0}
\definecolor{gray}{rgb}{0.5,0.5,0.5}
\definecolor{mauve}{rgb}{0.58,0,0.82}

\lstset{
	frame=single, 
	language=Python, 
	columns=flexible,
	basicstyle={\footnotesize\ttfamily},
	keywordstyle=\color{blue},
	commentstyle=\color{dkgreen},
	stringstyle=\color{red},
	title=\lstname
}


\renewcommand{\arraystretch}{1.3} % Changes the height of tables
\DeclareSIUnit\year{yr}
\DeclareSIUnit\parsec{pc}



\begin{document}

	\title[Photometry of M39]{Differential Photometry of M39}

	\author{Lucas, Miles}
	\author{Brandon, John}
	\affiliation{Iowa State University Department of Physics and Astronomy}

	\date{\today}


% Write here a short abstract (1 paragraph) describing what you achieved in this lab.

	\begin{abstract}


	\end{abstract}

	\maketitle
%________________________________________________________________________

%	Write here a summary of the goals of this lab. Be brief – do not exceed the end of the first page.

	\section{Introduction}


%________________________________________________________________________

%	Describe your telescope and camera setup. For the first lab you should describe the setup in more details. Subsequent labs can refer to the procedure described in the first lab, and focus more on the data acquisition itself. Important details to write include: which stars have been observed, which filters were used (BVI), what exposure times and what calibration procedures have been followed (e.g., dark frames acquisition, photometric standards observed, etc.). Mention also the observing conditions (weather, moon presence, temperature, etc.). You should include the observing log appendix A (a table or a scan of a hardcopy). This section can be as long as necessary, but you should strive for brevity (you can refer to the lab guide to avoid repeating all steps, but make sure you explain anything you did differently, problems encountered, etc.). Use tables to summarize information clearly.

	\section{Data Acquisition and Setup}
	Observations were made on \date{06 September 2017} at the Zaffarano Hall observation deck in Ames, Iowa. The night was mostly clear and the ambient temperature was around \SI{12}{\degreeCelsius}. The moon was full that night which caused higher than usual lunar presence. Observations were made using a Meade 8" reflector telescope with an SBIG ST-402ME CCD camera with internal V, B, and I filters. 
	
	Setting up the telescope was the same as previous observations made with the 8" Meade telescope at Zaffarano Hall. An obstacle we faced with alignment and slewing was the misalignment of our sight by a significant amount. To combat this, we shined a laser through the eyepiece to roughly show the target of the main mirror.
	
	We took 15 frames of data at \SI{13}{\second} at two locations in the sky. Of those 15, 5 were with photometric V, 5 were with photometric B, and 5 were dark frames. The first target contained M39 objects x3 and x5, however this data was deleted by accident and was not used for analysis. The second target contained objects x1, x4, x7, and x9, all of which are recorded in \autoref{table:log}. Of these images, another issue we encountered was centering the images as the telescope shifted according to its tracking movements. Because of this, a few of the images do not contain a clear view of M39 x4. 

%________________________________________________________________________

%Describe what you did to analyze the data. For the first lab this section will be minimal. For subsequent labs you should describe in detail the steps you did to analyze the data.  If applicable you can attach the code you used to analyze the data in Appendix B (either a log of your IDL section, or the code of any scripts you wrote). You can also attach screenshots of your computer session if it helps your discussion. Include tables of raw data in this section. The goal of this section is to describe unambiguously what you did to derive your final images and measured quantities from the raw data. Again be brief but complete (use as much space as you need).
	\section{Data Analysis}
	Our data analysis involved preparing our science images and performing differential photometry on them. To prepare our images we created a median dark frame in AstroImageJ for our only exposure time, \SI{13}{\second}. We then subtracted this frame from each science image to filter out systematic noise from our CCD. These images were then grouped into two stacks, one for each photometric filter.
	
	We used AstroImageJ to process each stack for the differential photometry. For use in the differential photometry algorithm we use the CCD reported electronic gain of \SI{1.49}{\elementarycharge\per{ADU}}. AstroImageJ processes the stack of images in a way that allows for quick calculations. An aperture can be placed on each star of interest in an image and AstroImageJ allows placing target stars and reference stars, where the reference stars have predetermined magnitudes.
	
	After placing the apertures for one image in the stack we move forward in the stack and by choosing the same initial aperture AstroImageJ will place the remaining apertures automatically. After each aperture has been placed on each image in the stack and the reference star magnitudes have been set, AstroImageJ will do the photometry and creates a measurement table with the results. This table is shown with the relevant information in \autoref{table:fullres}. This table shows results for target and reference stars and their names are shown in \autoref{fig:map} with reference to the object names of M39.


%________________________________________________________________________

%	Describe in detail the results you obtained from the quantitative analysis of your data, as explained in the lab guide.
	\section{Results}

	The table of the results from AstroImageJ are in \autoref{table:fullres} and the results calculated using \autoref{eqn:wa} and \autoref{eqn:ws} are in \autoref{table:phot}. 

	\begin{table}
		\centering
		\caption{Photometry Results}
		\begin{tabular*}{0.6\linewidth}{@{\extracolsep{\fill}}rrr}
			\hline
			Object &                 V Mag  &                 B Mag  \\ \hline\hline
			    x1 & \SI{7.7707\pm0.0039}{} & \SI{7.8078\pm0.0018}{} \\
			    x4 &                     NA &                     NA \\
			    x7 & \SI{9.3294\pm0.0045}{} &  \SI{9.473\pm0.0042}{} \\
			    x9 & \SI{9.9790\pm0.0035}{} &  \SI{10.233\pm0.011}{} \\ \hline
		\end{tabular*}
		\label{table:phot}
	\end{table}

%________________________________________________________________________

%	Add here anything else you want to say, and summarize your results. This section should be very brief, not more than a couple of paragraphs
	\section{Conclusions}


%________________________________________________________________________

	\section*{Acknowledgments}

	Thank you to Dr. Charles Kerton and Brandon Marshall for their guidance and assistance in this work.

%	\printbibliography


%________________________________________________________________________

	\onecolumngrid
	\appendix
	\section{Observation Log}

	\begin{table}[h!]
		\centering
		\caption{Observed 06 September 2017 by Miles Lucas and John Brandon}
\begin{tabular}{clclcccl}
	\hline
	Time  & File                   & N Frames & Object                                     & Filter &     Exposure     &       Camera Temp.        & Notes       \\ \hline\hline
	21:39 & M39\_2\_V\_13s\_       &    5     & M39 Objects x1, x4, x7, and x9; stars E, D &   V    & \SI{13}{\second} & \SI{5.33}{\degreeCelsius} &             \\
	21:41 & M39\_2\_V\_13s\_dark\_ &    5     & M39 Objects x1, x4, x7, and x9; stars E, D &   V    & \SI{13}{\second} & \SI{5.33}{\degreeCelsius} & Dark frames \\
	21:43 & M39\_2\_B\_13s\_       &    5     & M39 Objects x1, x4, x7, and x9; stars E, D &   B    & \SI{13}{\second} & \SI{5.33}{\degreeCelsius} &             \\ \hline
\end{tabular}

		\label{table:log}
	\end{table}
%________________________________________________________________________

	\section{Photometry Results}
	
	\begin{table}[h!]
		\centering
%		\input{results}
		\label{table:fullres}
	\end{table}


\end{document}
